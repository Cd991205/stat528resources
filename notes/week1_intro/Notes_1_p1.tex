% Options for packages loaded elsewhere
\PassOptionsToPackage{unicode}{hyperref}
\PassOptionsToPackage{hyphens}{url}
\PassOptionsToPackage{dvipsnames,svgnames,x11names}{xcolor}
%
\documentclass[
  ignorenonframetext,
]{beamer}
\usepackage{pgfpages}
\setbeamertemplate{caption}[numbered]
\setbeamertemplate{caption label separator}{: }
\setbeamercolor{caption name}{fg=normal text.fg}
\beamertemplatenavigationsymbolsempty
% Prevent slide breaks in the middle of a paragraph
\widowpenalties 1 10000
\raggedbottom
\setbeamertemplate{part page}{
  \centering
  \begin{beamercolorbox}[sep=16pt,center]{part title}
    \usebeamerfont{part title}\insertpart\par
  \end{beamercolorbox}
}
\setbeamertemplate{section page}{
  \centering
  \begin{beamercolorbox}[sep=12pt,center]{part title}
    \usebeamerfont{section title}\insertsection\par
  \end{beamercolorbox}
}
\setbeamertemplate{subsection page}{
  \centering
  \begin{beamercolorbox}[sep=8pt,center]{part title}
    \usebeamerfont{subsection title}\insertsubsection\par
  \end{beamercolorbox}
}
\AtBeginPart{
  \frame{\partpage}
}
\AtBeginSection{
  \ifbibliography
  \else
    \frame{\sectionpage}
  \fi
}
\AtBeginSubsection{
  \frame{\subsectionpage}
}
\usepackage{amsmath,amssymb}
\usepackage{lmodern}
\usepackage{iftex}
\ifPDFTeX
  \usepackage[T1]{fontenc}
  \usepackage[utf8]{inputenc}
  \usepackage{textcomp} % provide euro and other symbols
\else % if luatex or xetex
  \usepackage{unicode-math}
  \defaultfontfeatures{Scale=MatchLowercase}
  \defaultfontfeatures[\rmfamily]{Ligatures=TeX,Scale=1}
\fi
% Use upquote if available, for straight quotes in verbatim environments
\IfFileExists{upquote.sty}{\usepackage{upquote}}{}
\IfFileExists{microtype.sty}{% use microtype if available
  \usepackage[]{microtype}
  \UseMicrotypeSet[protrusion]{basicmath} % disable protrusion for tt fonts
}{}
\makeatletter
\@ifundefined{KOMAClassName}{% if non-KOMA class
  \IfFileExists{parskip.sty}{%
    \usepackage{parskip}
  }{% else
    \setlength{\parindent}{0pt}
    \setlength{\parskip}{6pt plus 2pt minus 1pt}}
}{% if KOMA class
  \KOMAoptions{parskip=half}}
\makeatother
\usepackage{xcolor}
\newif\ifbibliography
\setlength{\emergencystretch}{3em} % prevent overfull lines
\providecommand{\tightlist}{%
  \setlength{\itemsep}{0pt}\setlength{\parskip}{0pt}}
\setcounter{secnumdepth}{-\maxdimen} % remove section numbering
\usepackage{graphicx}
\usepackage{bm}
\usepackage{array}
\usepackage{amsmath}
\usepackage{amsthm}
\usepackage{amsfonts}
\usepackage{amssymb}
\usepackage{tikz-cd}
\usepackage{url}
\definecolor{foreground}{RGB}{255,255,255}
\definecolor{background}{RGB}{34,28,54}
\definecolor{title}{RGB}{105,165,255}
\definecolor{gray}{RGB}{175,175,175}
\definecolor{lightgray}{RGB}{225,225,225}
\definecolor{subtitle}{RGB}{232,234,255}
\definecolor{hilight}{RGB}{112,224,255}
\definecolor{vhilight}{RGB}{255,111,207}
\setbeamertemplate{footline}[page number]
\ifLuaTeX
  \usepackage{selnolig}  % disable illegal ligatures
\fi
\IfFileExists{bookmark.sty}{\usepackage{bookmark}}{\usepackage{hyperref}}
\IfFileExists{xurl.sty}{\usepackage{xurl}}{} % add URL line breaks if available
\urlstyle{same} % disable monospaced font for URLs
\hypersetup{
  pdftitle={STAT 528 - Advanced Regression Analysis II},
  pdfauthor={Generalized Linear Models},
  colorlinks=true,
  linkcolor={Maroon},
  filecolor={Maroon},
  citecolor={Blue},
  urlcolor={blue},
  pdfcreator={LaTeX via pandoc}}

\title{STAT 528 - Advanced Regression Analysis II}
\author{Generalized Linear Models}
\date{}
\institute{Daniel J. Eck\\
Department of Statistics\\
University of Illinois}

\begin{document}
\frame{\titlepage}

\begin{frame}
\newcommand{\R}{\mathbb{R}}
\newcommand{\Prob}{\mathbb{P}}
\newcommand{\Proj}{\textbf{P}}
\newcommand{\Hcal}{\mathcal{H}}
\newcommand{\rootn}{\sqrt{n}}
\newcommand{\p}{\mathbf{p}}
\newcommand{\E}{\text{E}}
\newcommand{\Var}{\text{Var}}
\newcommand{\Cov}{\text{Cov}}
\end{frame}

\begin{frame}{Agenda for today}
\protect\hypertarget{agenda-for-today}{}
\begin{itemize}
\tightlist
\item
  Course Syllabus and organization
\item
  Introductory materials
\item
  Course software and GitHub
\end{itemize}
\end{frame}

\begin{frame}{Materials and organization}
\protect\hypertarget{materials-and-organization}{}
\begin{itemize}
\tightlist
\item
  Install or update R and RStudio
\item
  Write homework assignments in RMarkdown and submit pdfs/html
\item
  Submit homework in your personal GitHub repository within my GitHub
  STAT 528 organization
\end{itemize}
\end{frame}

\begin{frame}{Computer resources}
\protect\hypertarget{computer-resources}{}
\begin{itemize}
\item
  The R Project for Statistical Computing:
  \url{https://www.r-project.org/}
\item
  RStudio as an integrated development environment for R:
  \url{https://www.rstudio.com/}
\item
  R Markdown: \url{https://rmarkdown.rstudio.com/}
\item
  The tidyverse for data science: \url{https://www.tidyverse.org/}
\item
  GitHub for version control: \url{https://github.com/}
\item
  UIUC CS GitHub repo creator tool:
  \url{https://wiki.illinois.edu/wiki/pages/viewpage.action?spaceKey=CSID&title=GitHub+repo+creator+tool}
\end{itemize}
\end{frame}

\begin{frame}{Course layout}
\protect\hypertarget{course-layout}{}
\begin{itemize}
\tightlist
\item
  Our course will start with a detailed treatment of exponential family
  theory
\item
  Motivation of GLMs from exponential family theory follows
\item
  We will then discuss GLMs for binary and count responses as well as
  multinomial regression
\item
  Next we will discuss data separation and GLM diagnostics
\item
  Spring break!
\item
  We will resume with contingency tables
\item
  The remainder of the course will be devoted to a detailed treatment of
  more advanced regression topics:

  \begin{itemize}
  \tightlist
  \item
    linear mixed-effects models
  \item
    generalized linear mixed-effects models and generalized estimating
    equations
  \item
    aster models for life history analysis
  \item
    multivariate regression and variance reduction via envelope
    methodology
  \end{itemize}
\end{itemize}
\end{frame}

\begin{frame}{Course layout (part II)}
\protect\hypertarget{course-layout-part-ii}{}
\begin{itemize}
\tightlist
\item
  The course will start with theory
\item
  It will then be a combination of methodology and data analysis
\item
  When we return from spring break, the course will be a mix of theory,
  methodology, and data analysis
\end{itemize}
\end{frame}

\begin{frame}{Student Learning Outcomes}
\protect\hypertarget{student-learning-outcomes}{}
Upon successful completion of this course students will be able to
conduct methodologically strong data analyses that can answer questions
of scientific interest.

Students will gain written communication skills, will be able to present
their data analyses in the form of a reproducible technical report, and
will gain experience with data science workflow.
\end{frame}

\begin{frame}{Background on topics}
\protect\hypertarget{background-on-topics}{}
\begin{itemize}
\tightlist
\item
  Distributions
\item
  Likelihoods
\item
  Exponential families
\item
  GLMs
\item
  LMMs
\end{itemize}

\vspace*{12pt}

There is more detail in the \texttt{introduction.pdf} notes. Some of
this additional detail is useful for your first homework assignment.
\end{frame}

\begin{frame}{Bernoulli and Binomial distributions}
\protect\hypertarget{bernoulli-and-binomial-distributions}{}
A random variable \(Y \sim\) Bernoulli\((p)\) has mass function \[
  f(y) = \left\{\begin{array}{cc}
    p   & y = 1 \\
    1-p & y = 0
  \end{array}\right.
\] where \(\text{E}(Y) = p\) and \(\text{Var}(Y) = 1 - p\), and
\(0 < p < 1\) is a success probability.

\vspace*{12pt}

A random variable \(Y \sim\) Binomial\((n,p)\) has mass function \[
  f(y) = {n \choose y}p^y(1-p)^{1-y}, \qquad y = 0,1,\ldots, n,
\] where \(\text{E}(Y) = np\) and \(\text{Var}(Y) = np(1-p)\). The
Binomial distribution arises as a sum of Bernoulli trials.
\end{frame}

\begin{frame}{Multinomial distribution}
\protect\hypertarget{multinomial-distribution}{}
\begin{itemize}
\tightlist
\item
  \(n\) independent trials
\item
  each trial results in one of \(c\) categories being observed
\item
  probability vector \(\mathbf{p}= (p_1,\ldots, p_c)\)
\item
  \(N_j\) be the total number of observations in category level
\end{itemize}

We will say \[
  (N_1,\ldots,N_c) \sim \text{multinomial}(n,\mathbf{p}) 
\] with mass function \[
  f(n_1,\ldots,n_c) = {n \choose n_1 \ldots n_c} p_1^{n_1}\cdots p_c^{n_c}, 
    \qquad \sum_{j=1}^c n_j = n,
\] where:

\begin{itemize}
\tightlist
\item
  \(\text{E}(N_j) = np_j\)
\item
  \(\text{Var}(N_j) = np_j(1-p_j)\)
\item
  \(\text{Cov}(N_j,N_k) = -np_jp_k\) where \(j \neq k\).
\end{itemize}
\end{frame}

\begin{frame}{Poisson distribution}
\protect\hypertarget{poisson-distribution}{}
A random variable \(Y \sim\) Poisson\((\mu)\), \(\mu > 0\), has mass
function \[
  f(y) = \frac{\mu^y e^{-\mu}}{y!},  \qquad y = 0,1,\ldots,
\] where \(\text{E}(Y) = \mu\) and \(\text{Var}(Y) = \mu\).
\end{frame}

\begin{frame}{Likelihoods}
\protect\hypertarget{likelihoods}{}
\begin{itemize}
\item
  For a model with parameter \(\theta\), the \textbf{likelihood}
  \(L(\theta)\) is the joint density of data at its observed values, as
  a function of \(\theta\).
\item
  When data is iid the likelihood and log likelihood \(l(\theta)\) will
  be \[
  L(\theta) = \prod_{i=1}^n f_\theta(y_i), \qquad 
    l(\theta) = \sum_{i=1}^n \log(f_\theta(y_1)).
  \]
\item
  A maximum likelihood estimate (MLE) \(\hat\theta\) maximizes
  \(l(\theta)\) and \(L(\theta)\). The estimate \(\hat\theta\) is
  usually the unique solution of
  \(\frac{\partial}{\partial\theta}l(\theta) = 0\).
\item
  \(\sqrt{n}(\hat\theta - \theta) \to N(0, I(\theta)^{-1})\) where \[
  I(\theta) = -\text{E}\left(\frac{\partial^2 l(\theta)}{\partial \theta^2}\right).
  \]
\end{itemize}
\end{frame}

\begin{frame}{Exponential family}
\protect\hypertarget{exponential-family}{}
An \emph{exponential family of distributions} is a parametric
statistical model having log likelihood that takes the form
\begin{equation} \label{expolog}
    l(\theta) = \langle y,\theta \rangle - c(\theta),
\end{equation} where:

\begin{itemize}
\tightlist
\item
  \(y\) is a vector statistic
\item
  \(\theta\) is a vector parameter
\item
  \(\langle y,\theta \rangle\) is the usual inner product
\item
  \(c(\theta)\) is the cumulant function.
\end{itemize}

We have: \begin{align*}
  \text{E}_\theta(Y) &= \nabla c(\theta), \\  
  \text{Var}_\theta(Y) &= \nabla^2 c(\theta).
\end{align*}

Examples: Binomial, Multinomial, Poisson, normal, etc.
\end{frame}

\begin{frame}{GLM}
\protect\hypertarget{glm}{}
Recall in LM: \[
   y = x^T\beta + \varepsilon, \qquad \text{E}(y|x) = x^T\beta
\]

\vspace*{12pt}

Generalized linear models (GLM) are extensions to the above where in
GLM: \[
 \text{E}_\theta(y_i|x_i) = g(x_i^T\beta) 
\] which implies that we can write \[
  g^{-1}\left(\text{E}_\theta(y_i|x_i)\right) = x_i^T\beta.
\] These models have very nice statistical properties when the
underlying model is an exponential family.
\end{frame}

\begin{frame}{LMM}
\protect\hypertarget{lmm}{}
The basic LMM takes the form \[
  Y = X\beta + Zb + \varepsilon \qquad \text{or} \qquad Y\mid b \sim N(X\beta + Zb, \sigma^2I),
\] where:

\begin{itemize}
\tightlist
\item
  \(Y\) is the response vector
\item
  \(X\) is a fixed-effects model matrix
\item
  \(\beta\) is a fixed-effects coefficient vector
\item
  \(Z\) is a model matrix of random-effects
\item
  \(b\) is a vector of random effects
\item
  \(\sigma^2\) is the variance of the error distribution.
\end{itemize}

If we further assume that \(b \sim N(0, \sigma^2D)\) then the
unconditional response \(Y\) is distributed as \[
  Y \sim N(X\beta, \sigma^2(I + ZDZ^T)).
\]
\end{frame}

\begin{frame}{Multivariate regression model}
\protect\hypertarget{multivariate-regression-model}{}
The basic multivariate regression model takes the form \[
  Y = \alpha +\beta X  + \varepsilon,
\] where:

\begin{itemize}
\tightlist
\item
  \(Y\) is the response vector
\item
  \(\alpha\) is an intercept vector
\item
  \(X\) is a predictor vector
\item
  \(\beta\) is a coefficient matrix
\item
  \(\varepsilon \sim N(0, \Sigma)\) where \(\Sigma > 0\).
\end{itemize}
\end{frame}

\end{document}
